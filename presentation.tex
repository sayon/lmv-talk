% !TEX TS-program = pdflatex
% !TEX encoding = UTF-8 Unicode

\documentclass{beamer}


\mode<presentation>
{
    %\usetheme{Warsaw}


}
\setbeamertemplate{frametitle}
{\begin{centering}\smallskip
\insertframetitle\par
\smallskip\end{centering}}
\setbeamertemplate{itemize item}{$\bullet$}
\setbeamertemplate{navigation symbols}{}
\setbeamertemplate{footline}[text line]{%
    \hfill\strut{%
        \scriptsize\sf\color{black!60}%
        \quad\insertframenumber
    }%
    \hfill
}

\usepackage[english]{babel}
% or whatever

\usepackage[utf8]{inputenc}
% or whatever		

\usepackage{times}
\usepackage[T1]{fontenc}


\usepackage{mathrsfs}
\usepackage{mathtools}
\usepackage{proof} 


%% Coq listings need it: %%
\definecolor{ltblue}{rgb}{0,0.4,0.4}
\definecolor{dkblue}{rgb}{0,0.05,0.4}
\definecolor{dkgreen}{rgb}{0,0.35,0}
\definecolor{dkviolet}{rgb}{0.3,0,0.5}
\definecolor{dkred}{rgb}{0.5,0,0}
\definecolor{grey}{rgb}{0.5,0.5,0.6}
\usepackage{listings}
\usepackage{lstcoq}
\lstset{
    basicstyle=\footnotesize,
    columns=fullflexible, 
language=Coq} 

%settings 
\everymath{\displaystyle}

%--------

\title{Coq: the reflection principle\\and encoding of mathematical hierarchies with canonical instances}

\author{I.Zhirkov}


\date{2016}
\subject{Theoretical Computer Science}


% Delete this, if you do not want the table of contents to pop up at
% the beginning of each subsection:
\AtBeginSubsection[]
{
    \begin{frame}<beamer>{Outline}
        \tableofcontents[currentsection,currentsubsection]
    \end{frame}
}


% If you wish to uncover everything in a step-wise fashion, uncomment
% the following command: 

%\beamerdefaultoverlayspecification{<+->}


\begin{document}

\begin{frame}
    \titlepage
\end{frame}


\begin{frame}{What is Coq?}
    \begin{itemize}
        \item Functional programming language with dependent types
            \pause
            \begin{itemize}
                \item one type can depend from others;
                \item function types can have dependencies;
                \item function can accept types.
            \end{itemize}
            \pause
        \item Logical system (intuitionistic logic)
            \begin{itemize}
                \item types and programs can be used to encode propositions and
                    proofs. 
            \end{itemize}
    \end{itemize}
\end{frame}

\begin{frame}{As a programming language}
    \begin{itemize}
        \item A variation of a typed lambda calculus
        \item Usual stuff:
            \begin{itemize}
                \item Functions;
                \item Types and pattern matching;
                \item Limited recursion (can't encode infinite computations).
            \end{itemize}
    \end{itemize}
\end{frame}


\begin{frame}{As a logical system}
    
\begin{itemize}
    \item Intuitionistic $\sim$ Constructive
        \begin{itemize}
            \item No $\top$ or $\bot$ values, assigned to each formula;
            \item We infer proofs instead.
        \end{itemize}
    \end{itemize}
\end{frame}

\begin{frame}[fragile]{Correspondance}
    \begin{itemize}
        \item \textbf{Axiom} (provable apriori) = Axioms, Selected type inhabitants

            \begin{lstlisting}
Inductive nat := 
| Zero: nat  (* <- witness *)
| Succ: nat -> nat.

Axiom em: forall A, A \/ not A.
            \end{lstlisting}

        \item \textbf{Theorem with premises} = encode a proof algorithm.
    
            Proof of $A \land B \rightarrow C$: 
            ``Given a proof for  ($A$ and $B$) build one for $C$''
        \item \textit{Modus ponens} is an application.
    \end{itemize}

\end{frame}

\begin{frame}[fragile]{Example of a dependent type}
    Encode the existence quantifier:
\begin{lstlisting}
Inductive sig (A : Type) (P : forall _ : A, Prop) : Type :=
    exist : forall (x : A) (_ : P x), sig P
\end{lstlisting}

A little less formal:
$$ \bigg( a: A, (fun\ x:A => \_ : Prop) \ a ) \bigg) $$
\end{frame}


\begin{frame}[fragile]
    Data is decomposed via induction (relevant axioms are generated automatically).

\begin{lstlisting}
Inductive nat := 
| Zero: nat  (* <- witness *)
| Succ: nat -> nat.

Check nat_ind.
(* :> *)
nat_ind: forall P : nat -> Prop,
          P Zero 
          -> (forall n : nat, P n -> P (Succ n)) 
          -> forall n : nat, P n
          
\end{lstlisting}
\end{frame}


\begin{frame}[fragile]{Equality in Coq}

\begin{itemize}
    \item Definitional (judgemental)
    \item Propositional
\begin{lstlisting}
Inductive eq (A : Type) (x : A) :=  eq_refl : x = x.
\end{lstlisting}
    \item Computable -- if we define it (not part of the system)
\end{itemize}
\end{frame}

\begin{frame}[fragile]{Examples}
    ``Data'' lives in \coqe{Set}, Propositions live in \coqe{Prop}.

    \lstinputlisting{test.v}
\end{frame}

\begin{frame}[fragile]{Examples}
    \lstinputlisting{proofterm_ex1.v}
\end{frame}

\begin{frame}[fragile]{Examples: tactics}
    Big proof terms are hard to write at once. 

    Adapt an iterative approach of their construction.

    \lstinputlisting{proofterm_ex1.v}
    
    =>
   
    \lstinputlisting{proofterm_ex2.v}

\end{frame}

\begin{frame}[fragile]{Examples: tactics}
    \begin{lstlisting}
    (* This definition is not complete!*)
    Definition t1 : (A -> B -> C) -> A -> B -> C. 
    \end{lstlisting}
    Goal (shows current step in proof construction: what we want to obtain and what we have in context):

    \begin{lstlisting}
1 subgoal, subgoal 1 (ID 1)

     (A -> B -> C) -> A -> B -> C

    \end{lstlisting}

\end{frame}

\begin{frame}[fragile]{Examples: tactics}
    \begin{lstlisting}
Definition t1 : (A -> B -> C) -> A -> B -> C.
  intros f a b.
  \end{lstlisting}
  Goal:

    \begin{lstlisting}
    1 subgoal, subgoal 1 (ID 4)

        f : A -> B -> C
        a : A
        b : B
        ---
        C
    \end{lstlisting}

\end{frame}

\begin{frame}[fragile]{Examples: tactics}
    
   \begin{lstlisting}
Definition t1 : (A -> B -> C) -> A -> B -> C.
  intros f a b.
  apply f.
    \end{lstlisting}

    Goal (2 subgoals because $f$ has two arguments of type $A$ and $B$ ).

    \begin{lstlisting}
    2 subgoals, subgoal 1 (ID 5)

        f : A -> B -> C
        a : A
        b : B
        --- 
        A

    subgoal 2 (ID 6) is:
        B
    \end{lstlisting}

\end{frame}


\begin{frame}[fragile]{Examples: tactics}
    \begin{lstlisting}
Definition t1 : (A -> B -> C) -> A -> B -> C.
  intros f a b.
  apply f.
  exact a.
    \end{lstlisting}
    Goal (one subgoal out):

    \begin{lstlisting}
    1 subgoal, subgoal 1 (ID 6)

        f : A -> B -> C
        a : A
        b : B
       --- 
        B

    \end{lstlisting}

\end{frame}

\begin{frame}[fragile]{Examples: tactics}
    \begin{lstlisting}
Definition t1 : (A -> B -> C) -> A -> B -> C.
  intros f a b.
  apply f.
  exact a.
  exact b.
    \end{lstlisting}
We are done here.


\end{frame}

\begin{frame}
    \begin{itemize}
        \item Some things are not decidable, so in general we can't autocompute proofs.

            Explicitly provide proof of primarity
        \item But some things are decidable and we want to use it.
    \end{itemize}

    Unleash bruteforce computing to spare human time.
\end{frame}


\begin{frame}{Reflection}

    is a mean to switch between constructive reasoning and a computable 
    (via beta reductions) form of a proposition.
    
    \emph{Entirely inside Coq's theory.}

    \texttt{reflect1.v} 

    It is a fondament of an \textbf{ssreflect} library

\end{frame}

\begin{frame}{ssreflect}
    \begin{itemize}
    \item Part of Mathematical Components library (used to formalize Odd Order
        and Four color theorems)
    \item Extends Coq with a minimalistic proof language (enough for most proofs)
    \end{itemize}

\end{frame}

\begin{frame}{ssreflect}
    \begin{itemize}
    \item \emph{move} manipulates context and goal;
    \item \emph{case} performs case analysis on a term;
    \item \emph{elim} performs induction;
    \item \emph{rewrite} performs a rewrite using a hypothesis $A = B$ (augmented);
    \item \emph{apply} applies a function;
    \item \emph{done} tries to finish proof automatically.
    \end{itemize}

\end{frame}

\begin{frame}[fragile]{ssreflect tactics}

    \begin{lstlisting}
move=> h.
    \end{lstlisting}

    $\Gamma$ |-  $H \rightarrow X $
    
===

$\Gamma \cup \{ h: H \}$ |- $X$ 
\end{frame}

\begin{frame}[fragile]{ssreflect tactics}

    \begin{lstlisting}
    move: h.
    \end{lstlisting}

    $\Gamma \cup \{ h:H \} $ |-  $X $
    
===

$\Gamma $ |- $ H \rightarrow X$ 
\end{frame}

\begin{frame}[fragile]{ssreflect tactics}

    \begin{lstlisting}
    move /h.
    \end{lstlisting}

    $\Gamma \cup \{ h:H \} $ |-  $ a -> X $
    
===

$\Gamma \cup \{ h:H \} $ |- $ h \ a \rightarrow X$ 
\end{frame}

\begin{frame}[fragile]{Many, many ways to combine}
    \begin{lstlisting}
    move=> H [Hl|[Hrl|Hrr]].
    elim: x y => //=.
    move /eqP => -> [] [] =>/=.  

    (* Real world proof *)
    Lemma streeR_total: total streeR. 
    by move: HRtotal;
    rewrite /total /streeR /treeR => ?[[??]?][[??]?] => //=.
    Qed.
    \end{lstlisting}
\end{frame}

\begin{frame}{What does ssreflect reflect?}
    \begin{itemize}
        \item Equality;
        \item Has (a list has an item...)
        \item Logical or/and/not/implication...
        \item Many more decidable things
    \end{itemize}
    
    \texttt{reflect1.v}
\end{frame}

\begin{frame}

    An understanding of canonical structures is required to understand how to
produce ``general equalities'' etc. 

\end{frame}

\begin{frame}{Canonical instances}
    \begin{itemize}
        \item A way to encode proof search;
        \item A database of hints for type solver;
        \item Currenty only for specific cases of dependent records.
    \end{itemize}
    \texttt{canonical1.v}
\end{frame}




\begin{frame}{Canonical instances}
    \begin{itemize}
        \item Encode mathematical structures' hierarchy (like semigroups,
        monoids etc).
        \item No strict inheritance
        \item Dualism between coercions and CI (we can always coerce to a 
            simpler form, but how to select the correct rich form? )

        \item Example: model general equality and ordering separately and make
    them usable for a type.  \end{itemize}
    \texttt{eq\_leq.v}

\end{frame}

\begin{frame}{Similar mechanisms}
    \begin{itemize}
        \item Type Classes 
        \item Canonical Structures (a bit more general)
        \item Unification Hints (more general)
    \end{itemize}

\end{frame}



\begin{frame}{Readings on ssreflect, reflection and canonical structures}
    \begin{itemize}
        \item ssreflect docs 
        \item ssreflect tutorial by G. Gontier et al
        \item Ilya Sergey ``Programs and proofs'' (best ssreflect book
        available) 
        \item Assia Mahboubi, Enrico Tassi ``Canonical Structures
            for the working Coq user'' 
    \end{itemize}
\end{frame}

\begin{frame}
    Thank you.

    Question time.
\end{frame}

\end{document}


